\documentclass{article}
%% course name and homework number
\newcommand{\coursename}{Computer Network}
\newcommand{\hwnumber}{6}
\usepackage[utf8]{inputenc}
\usepackage{amsmath}
\usepackage{amssymb}
\usepackage{amsfonts}
\usepackage{amssymb}
\usepackage{minted}
\usepackage{url}
\usepackage{graphicx}
\usepackage{algorithm}
\usepackage{booktabs}
\usepackage{tabularx}
\usepackage{lipsum}
\usepackage{algorithmic}
\graphicspath{ {img/} }
\usepackage{titlesec}
\usepackage[a4paper,margin=1in,footskip=0.25in]{geometry}
\usepackage{fancyhdr}
\usepackage{enumitem} %% custom auto-index for enumerate environment
\pagestyle{fancy}
\usepackage{hyperref}
\usepackage[%
  autocite    = superscript,
  backend     = bibtex,
  sortcites   = true,
  style       = numeric,
  ]{biblatex}
\addbibresource{hw6.bib}
%basic page layout

%draw finite state machine
\usepackage{tikz}
\newcommand{\Lcvy}{\mathcal{L}}
%header and footer settings
\lhead{\coursename \thinspace Assignment \hwnumber}
\chead{Yiping Deng}
\rhead{\today}

\titlelabel{\thetitle\enspace}

\begin{document}
\title{\coursename \thinspace Assignment \hwnumber}
\author{Yiping Deng}
\maketitle
\thispagestyle{fancy}

%% start of homework
\section*{Problem 6.1}
\subsection*{a)}
There are several steps concerning the lookup of
\verb+grader.eecs.jacobs-university.de+.
We need to lookup the AAAA records for IPv6 records.
We use the following dig command to get the result
\begin{verbatim}
dig +trace grader.eecs.jacobs-university.de AAAA
\end{verbatim}
and we will end up with the following result
\begin{verbatim}

; <<>> DiG 9.10.6 <<>> +trace grader.eecs.jacobs-university.de AAAA
;; global options: +cmd
.			452315	IN	NS	a.root-servers.net.
.			452315	IN	NS	e.root-servers.net.
.			452315	IN	NS	b.root-servers.net.
.			452315	IN	NS	j.root-servers.net.
.			452315	IN	NS	c.root-servers.net.
.			452315	IN	NS	h.root-servers.net.
.			452315	IN	NS	i.root-servers.net.
.			452315	IN	NS	k.root-servers.net.
.			452315	IN	NS	g.root-servers.net.
.			452315	IN	NS	f.root-servers.net.
.			452315	IN	NS	d.root-servers.net.
.			452315	IN	NS	l.root-servers.net.
.			452315	IN	NS	m.root-servers.net.
.			489115	IN	RRSIG	NS 8 0 518400 20190522050000 20190509040000 25266 . yBg5V5MdhPQcGc+FYRa9c7Jxh1tKSrrejQOjbRboWJBOz3+x4GSnS5Aw 1ec77ZhE2GAwLUV9rftRjCfO4R3E2yJ28ht5d5S5KYE2pcmQBaVg7Y21 K5hY/rUqxdH2sTHPK+zPTHX0cNHfVdbBPmzrH1XRShdTA+/Goa9gN2e2 LSUjmS+MQKneTzV+e0gfhbsJS0Qd/pI1RKBofdlDGPDuVSU+EUzB9uI1 0ZYUqVpfls5N/2Stu2QAVKtj2iDY91TI8RxWM9dQurbudFmvBGzkVtDh HzfkZRA/MDF3CaYXugTE003KbQBcppP6K7pM4IlPo+zzgHSyOQ3hw/Tc BqTalg==
;; Received 1097 bytes from 10.70.0.20#53(10.70.0.20) in 1 ms

de.			172800	IN	NS	s.de.net.
de.			172800	IN	NS	l.de.net.
de.			172800	IN	NS	z.nic.de.
de.			172800	IN	NS	n.de.net.
de.			172800	IN	NS	f.nic.de.
de.			172800	IN	NS	a.nic.de.
de.			86400	IN	DS	39227 8 2 AAB73083B9EF70E4A5E94769A418AC12E887FC3C0875EF206C3451DC 40B6C4FA
de.			86400	IN	RRSIG	DS 8 1 86400 20190522160000 20190509150000 25266 . b25ARMch60l7X6xtCtG3Z2zrDWtenl/WDGHhmhYr1hhjWvb3Tw4HNqUp 6fdEXHuxLI/VZHDl58D8mUfZgMwLHOFPqfsco/OXo1ANTB0ZRZGiFeGW 9EcMt9owgXwNIWh3KeOkcS3WipS5YQZZE6IZ+liiKBajKFiIfxYEYU5r FeHI2ONSxC1u1mj8cO01uJ9Z0bjjhL389dOMxWycF6bGounLVF/dIuEq yIDM7f3UX0A4/2fv+/Fp4MJSQRNsAMKqNfpqyqlhtWenfF/dvD9SPjdG ZfwP6lajUVrlmN/IPZRsx2NRG3IyrtstSClpnc440g1agtb4tClkTzoR fOIYbw==
;; Received 738 bytes from 199.9.14.201#53(b.root-servers.net) in 125 ms

jacobs-university.de.	86400	IN	NS	dns.iu-bremen.de.
jacobs-university.de.	86400	IN	NS	www.jacobs-utils.de.
H319DM5GC3EDEK691VQBHEHOT7VGGJ2B.de. 7200 IN NSEC3 1 1 15 BA5EBA11 H31EGRUDRBMFSM3HAQ6AMG96SJB4QAVI  NS SOA RRSIG DNSKEY NSEC3PARAM
H319DM5GC3EDEK691VQBHEHOT7VGGJ2B.de. 7200 IN RRSIG NSEC3 8 2 7200 20190516190105 20190509190105 26298 de. iFw2XlWZvYckjrvDHAdsHCWTeFKL8VsfIyx/kDGL5AWZETOGXghrz4Bs 6Bit24P9Q9Dm1OlrIKvNz0xwJyJA7puvkpEY1+jfOM8r34fD/1TxhDIn t5oXQkgk/VYrRQKJteQwM3h4ZfTH3Bt1gNnJFr052EI6VUno4iZoSs7N k4c=
SFAC58VBFNB14JPD7N2H5MOLE1O2L213.de. 7200 IN NSEC3 1 1 15 BA5EBA11 SFAEPI3LFEOB8NDDLF1S5816FTJPO3AS  A RRSIG
SFAC58VBFNB14JPD7N2H5MOLE1O2L213.de. 7200 IN RRSIG NSEC3 8 2 7200 20190516190105 20190509190105 26298 de. cFHmSGtb2OXMZ4r2lGDuIM4IniaHLDpiZ0HG5XJq/sCLYiRMRYYa8P9T XGf+QtgFVnVFOZdw5WkgFL2TOewB3iaR2GrZNRmUlt1VmVg7MtQeefAt lkPuhwc/s07qwJT72+RtnQI+aIr4S+j+41ol4d+qIEi4IvfNXnR8ZUZx TGc=
;; Received 611 bytes from 195.243.137.26#53(s.de.net) in 20 ms

grader.eecs.jacobs-university.de. 3600 IN CNAME	cantaloupe.eecs.jacobs-university.de.
cantaloupe.eecs.jacobs-university.de. 3600 IN AAAA 2001:638:709:3000::29
eecs.jacobs-university.de. 3600	IN	NS	ns1.ibr.cs.tu-bs.de.
eecs.jacobs-university.de. 3600	IN	NS	dns.jacobs-university.de.
eecs.jacobs-university.de. 3600	IN	NS	ns.eecs.jacobs-university.de.
;; Received 240 bytes from 212.201.44.22#53(dns.iu-bremen.de) in 3 ms


\end{verbatim}
The lookup process are\autocite{cloudflareDNS}:
\begin{enumerate}
\item
  A DNS query is fired to our DNS recursivse resolver provided via DHCP. In this case, it is
  \verb+10.70.0.20+. The query asked for AAAA records for
  \verb+grader.eecs.jacobs-university.de+.
\item
  The local DNS resolve on the network returns 13 root server to us. We use one
  of the root server \verb+g.root-servers.net+ to proceed with the query of
  \verb+de.+
\item
  we got 6 nameserver responsible for \verb+de.+ top level domains, we choose
  \verb+s.de.net+ to proceed with the next subdomain
  \verb+jacobs-university.de.+.
\item
  It gave us two nameserver,
  \verb+www.jacobs-utils.de.+
  and \verb+dns.iu-bremen.de.+, we choose
  \verb+www.jacobs-utils.de.+ to proceed with next level of subdomain,
  \verb+eecs.jacobs-university.de.+
\item
  We receive 3 nameservers at this level, then we proceed with
  \verb+dns.jacobs-university.de+ to finally lookup the AAAA record for
  \verb+grader.eecs.jacobs-university.de.+.
\item
  \verb+dns.jacobs-university.de+ found out that
  \verb+grader.eecs.jacobs-university.de.+ contains a CNAME record
  pointing to \verb+cantaloupe.eecs.jacobs-university.de.+. In the same
  response, we received AAAA records
  \verb+cantaloupe.eecs.jacobs-university.de. 3600 IN AAAA 2001:638:709:3000::29+
  together with additional information. Domain name lookup is finally complete.
\end{enumerate}
\subsection*{b)}
\label{subsec:srv}
SRV resoure record follows the following format\autocite{RFC2782}:
\begin{verbatim}
_Service._Proto.Name TTL Class SRV Priority Weight Port Target
\end{verbatim}
where
\begin{itemize}
\item
  \emph{Service} denotes the symbolic name of desired service, and an underscore is
  prepended to avoid the collision with natural DNS labels, and it is case insensitive.
\item
  \emph{Proto} denotes the symbolic name of the protocol, with a underscore
  prepended. \verb+_TCP+ and \verb+_UDP+ are the two most commonly used
  protocols. It is alsos case insensitive.
\item
  \emph{Name} denotes domain name it refers to (will be valid within).
\item
  \emph{TTL} denotes common DNS TTL.
\item
  \emph{Class} denotes common DNS Class.
\item
  \emph{Prority} denotes the priority of this target host.
\item
  \emph{Weight} denotes the relative weight for entries with the same priority.
\item
  \emph{Port} denotes the port number of the service.
\item
  \emph{Target} denotes the domain name of the the host providing the service.


\end{itemize}

It is designed to ``be used by clients
   for applications where the relevant protocol specification indicates
   that clients should use the SRV record.''\autocite{RFC2782}. It is commonly used
   to identify servers that host specfic service with certain protocol. It is
   properly defined in RFC2782, and it is stored in normal zone file.


   For example,
\begin{verbatim}
_ldap._tcp.example.net. 3600 IN	  SRV  0 0 389 phoenix.example.net.
\end{verbatim}
   denotes that it is serving LDAP on TCP protocol, with port number 389, 0
   priority and 0 weight and the server is hosted on \verb+phoenix.example.net.+.

   Prority and weight are used in totally different way. Firstly, ``a client \textbf{must} attempt to contact the target host with the lowest-numbered priority it
   can reach.'' When 2 entries have the same priority, the one with
   \textbf{higher} weight should have proportionally higher probability to be selected.
\subsection*{c)}
Using SRV resoruce record is a debatable topic. There are several pros and cons
for such matters.
\begin{table}[H]
\begin{tabularx}{\linewidth}{>{\parskip1ex}X@{\kern4\tabcolsep}>{\parskip1ex}X}
\toprule
\hfil\bfseries Pros
&
\hfil\bfseries Cons
\\\cmidrule(r{3\tabcolsep}){1-1}\cmidrule(l{-\tabcolsep}){2-2}

SRV is designed in a way that we can easily achieve client-side load balancing.
This can save a lot of cost for infrastructures cost for website owners.

SRV resource record on HTTP will allow websites to be served on any ports, not
just port 80. As some ISP will block 80 to save upbound traffic, it is
particularly useful to serve website on home ISP network. It can also bypass
certain censorship on the network and firewalls.

HTTP SRV resource record can redirect traffic to different domains under
different port numbers, allowing higher configurability in web services.

For distributed system, SRV resource record can be used for service discoverery.
This is useful as we are in the age of microservice architecture, and service
discoverery is the key to microservice architecture.

&

Using SRV might end up with non-deterministic visit to the host to open a
website. It is hard to aggregate access information and statistics, and it is
hard to pinpoint failure servers. The load balancing provided through SRV will
lead to security issue. As the internal infrastructures are exposed, targeting a
single server has never been easier.

The idea of serving HTTP on any ports via SRV resource record is terrifying.
Those HTTP traffic might bypass firewalls, leading to security vulnerability.
Cutting off traffic for HTTP server will be much hard. Imagine the case of a
child pornographic website serving using SRV resource record, port blacklist
will not be effective anymore.

The idea of redirect traffic to any server on any port can be abused and used to
launch denial of service attack.

Next level of DNS spoofing can be achieved. You not only are able to redirect to
different addresses, you can even redirect to different port number.


\\\bottomrule
\end{tabularx}
\caption{Pros and cons of using SRV resource record for HTTP}
\end{table}

\subsection*{d)}
As the Internet grow and introduction of IPv6, as well as the invention of
stricter security measures with DNS, more and more bytes are stuffed in DNS
messages. However, traditional DNS messages have a upper bound for the size--512
bytes. What's more, previous DNS protocol doesn't contain a mechanism to
advertise the capabilities to others. Certainly, this protocol is no longer
suitable for the development and the scalability of the Internet.

Hence, in RFC6891, EDNS0 is introduced. ``E'' stands for extension mechanisms.
It provides DNS messages with the capability to extend beyond the previous size
limit, and it provides extra data space for flags and return codes.

In the OPT resource record, CLASS field is used to store the requester's UDP
payload size. It denotes the number of bytes of the largest UDP payload that can
be delivered in the requester's network stack. TTL field is used to store
extended RCODE, version and flags. Extended RCODE is 8 bits, used together with the
original RCODE. It also contains a version field indicating the lowest
implemented level, and full support of RFC6891 is indicated by '0'. There are 2
flags supported, DO flags indicate the usage of DNSSEC. Z flags are usually 0,
and are designed for later usage.
\subsection*{e)}
We are using dig to find all the different A and AAAA records, and here are the
result for amazon.com:
\begin{table}[H]
  \centering
\begin{tabular}{|c|c|c|c|}
  \hline
  DNS address & Provider & A & AAAA \\
  \hline
  1.1.1.1 & Cloudflare & 176.32.103.205 & N/A \\
              & & 205.251.242.103  & \\
  & & 176.32.98.166  & \\
  \hline
  8.8.8.8 & Google & 205.251.242.103 & N/A \\
              & & 176.32.103.205 & \\
              & & 176.32.98.166 & \\
  \hline
  9.9.9.9 & Quad9 & 205.251.242.103 & N/A \\
              & & 176.32.98.166 & \\
  & & 176.32.103.205 & \\
  \hline

\end{tabular}
\caption{Amazon DNS records on public nameservers}
\end{table}
We can see that amazon.com does not support IPv6 yet, which is quite sad. They
always have the same answer, quite boring.

Let's test it again on Google.com:
\begin{table}[H]
\centering

\begin{tabular}{|c|c|c|c|}
  \hline
  DNS address & Provider & A & AAAA \\
  \hline
  1.1.1.1 & Cloudflare & 172.217.22.14 & 2a00:1450:4001:815::200e \\
  \hline
  8.8.8.8 & Google & 172.217.16.206 & 2a00:1450:4001:821::200e \\
  \hline
  9.9.9.9 & Quad9 & 172.217.22.46 & 2a00:1450:4001:821::200e \\
  \hline
\end{tabular}
\caption{Google DNS records on public nameservers}
\end{table}
This time, clearly we can see IPv4 address are different for every server, and
IPv6 is different for Cloudflare. As we observe from the IPv6, Cloudflare and
Quad9 address are in the same class C subnet, where the Google server report a
server in a differnet class C subnet.


\section*{Problem 6.2}
\subsection*{a)}
Multicast DNS provides the capabilities to look up DNS resource record in
absence of conventional managed DNS server, and it designs ``designates a portion
   of the DNS namespace to be free for local use''\autocite{RFC6762}.
There are three primary benefit of mDNS:
\begin{enumerate}
\item It requires little or no administration or configuration to setup.
\item It works with no extra infrastructure.
\item It works on infrastructure failures.
\end{enumerate}

Multicast DNS name will have the following format\autocite{RFC6762}:
\begin{verbatim}
single-dns-label.local.
\end{verbatim}

This protocol is defined in RFC6762\autocite{RFC6762}.

It depart from regular DNS protocol semantics in several ways:
\begin{enumerate}
\item It sends requests to multicast addresss \verb+224.0.0.251:5353+ (or
  \verb+[FF02::FB]:5353+ in IPv6), and the response is normally in the same
  multicast fashion. The response have a random delay of up to 500ms to avoid
  collisions with other responder, and it doesn't have a question section.
\item
  It then pickup the first response and resolve the addresses.
\item
  It can also operate on continuous querying, where the querying operation
  continues until no further responses are required, depending on the type of
  operation being performed\autocite{RFC6762}.
\item
  The TTL values are respected in a probabilistic way. ``To avoid the case where multiple Multicast DNS queriers on a network
   all issue their queries simultaneously, a random variation of 2\% of
   the record TTL should be added, so that queries are scheduled to be
   performed at 80-82\%, 85-87\%, 90-92\%, and then 95-97\% of the TTL.''\autocite{RFC6762}
\end{enumerate}


\subsection*{b)}
DNS-based service discovery is defined in RFC6763. It specifies ``how DNS resource records are named and
   structured to facilitate service discovery''\autocite{RFC6763}. It allows
   client to discover a list of named instances with certain service using DNS
   queries.

   Two types of DNS records can be used to facilitate service discovery--SRV and
   TXT records.

   SRV record, for service discovery, has the following form
   ``\verb+<Instance>.<Service>.<Domain>+''\autocite{RFC6763} and it provides the target
   host and port where it can reach. Example has been given above
   (\autoref{subsec:srv}).

   TXT record, in the other hand, has higher flexibility. It is stored as
   key/value pairs. It can provide more
   information than just IP addresses and port numbers. For example, a file
   server might have different volumns, and those information can be conveyed in
   the TXT record. Client fires a DNS PTR record query, and it will receive a
   set of zero or more names, which are the names of the DNS SRV/TXT record
   pairs.
   Even if all the information are in the SRV record, a TXT record is needed
   (empty record here).
\printbibliography
\end{document}
