\documentclass{article}
%% course name and homework number
\newcommand{\coursename}{Computer Network}
\newcommand{\hwnumber}{3}
\usepackage[utf8]{inputenc}
\usepackage{amsmath}
\usepackage{amssymb}
\usepackage{amsfonts}
\usepackage{amssymb}
\usepackage{minted}
\usepackage{graphicx}
\usepackage{algorithm}
\usepackage{algorithmic}
\usepackage[]{caption}
\graphicspath{ {img/} }
\usepackage{titlesec}
\usepackage[a4paper,margin=1in,footskip=0.25in]{geometry}
\usepackage{fancyhdr}
\usepackage{enumitem} %% custom auto-index for enumerate environment
\usepackage{url}
\pagestyle{fancy}
%basic page layout

%draw finite state machine
\usepackage{tikz}
\newcommand{\Lcvy}{\mathcal{L}}
%header and footer settings
\lhead{\coursename \thinspace Assignment \hwnumber}
\chead{Yiping Deng}
\rhead{\today}

\titlelabel{\thetitle\enspace}

\begin{document}
\title{\coursename \thinspace Assignment \hwnumber}
\author{Yiping Deng}
\maketitle
\thispagestyle{fancy}

%% start of homework
\section*{Problem 3}
\subsection*{a)}
Using topology defined in Listing \ref{listing:p3a}, we can execute the ping and
traceroute in mininet and verify the topology:
\captionof{listing}{Mininet output for a)}
\begin{minted}[linenos=true, frame=lines]{bash}
mininet> h1 ping6 2001:638:709:b::1
PING 2001:638:709:b::1(2001:638:709:b::1) 56 data bytes
64 bytes from 2001:638:709:b::1: icmp_seq=1 ttl=62 time=0.125 ms
64 bytes from 2001:638:709:b::1: icmp_seq=2 ttl=62 time=0.060 ms
64 bytes from 2001:638:709:b::1: icmp_seq=3 ttl=62 time=0.060 ms
^C
--- 2001:638:709:b::1 ping statistics ---
3 packets transmitted, 3 received, 0% packet loss, time 2050ms
rtt min/avg/max/mdev = 0.060/0.081/0.125/0.032 ms


mininet> h2 ping6 2001:638:709:a::1
PING 2001:638:709:a::1(2001:638:709:a::1) 56 data bytes
64 bytes from 2001:638:709:a::1: icmp_seq=1 ttl=62 time=0.058 ms
64 bytes from 2001:638:709:a::1: icmp_seq=2 ttl=62 time=0.062 ms
64 bytes from 2001:638:709:a::1: icmp_seq=3 ttl=62 time=0.073 ms
64 bytes from 2001:638:709:a::1: icmp_seq=4 ttl=62 time=0.064 ms
64 bytes from 2001:638:709:a::1: icmp_seq=5 ttl=62 time=0.060 ms
^C
--- 2001:638:709:a::1 ping statistics ---
5 packets transmitted, 5 received, 0% packet loss, time 4087ms
rtt min/avg/max/mdev = 0.058/0.063/0.073/0.008 ms



mininet> h1 traceroute 2001:638:709:b::1
traceroute to 2001:638:709:b::1 (2001:638:709:b::1), 30 hops max, 80 byte packets
 1  2001:638:709:a::f (2001:638:709:a::f)  0.033 ms  0.018 ms  0.008 ms
 2  2001:638:709:f::2 (2001:638:709:f::2)  0.018 ms  0.009 ms  0.009 ms
 3  2001:638:709:b::1 (2001:638:709:b::1)  0.019 ms  0.010 ms  0.010 ms



mininet> h2 traceroute 2001:638:709:a::1
traceroute to 2001:638:709:a::1 (2001:638:709:a::1), 30 hops max, 80 byte packets
 1  2001:638:709:b::f (2001:638:709:b::f)  0.030 ms  0.007 ms  0.007 ms
 2  2001:638:709:f::1 (2001:638:709:f::1)  0.019 ms  0.009 ms  0.008 ms
 3  2001:638:709:a::1 (2001:638:709:a::1)  0.019 ms  0.010 ms  0.010 ms
\end{minted}

\captionof{listing}{Basic topology network}
\label{listing:p3a}
\inputminted[linenos=true, frame=lines]{python}{src/p3.a.py}

\subsection*{b)}
Using the modified Listing \ref{listing:p3b}, we can use the traceroute to show
the asymetric path:
\captionof{listing}{Traceroute of b)}
\begin{minted}[linenos=true, frame=lines]{bash}
mininet> h1 traceroute 2001:638:709:b::1
traceroute to 2001:638:709:b::1 (2001:638:709:b::1), 30 hops max, 80 byte packets
 1  2001:638:709:a::f (2001:638:709:a::f)  0.042 ms  0.009 ms  0.006 ms
 2  2001:638:709:f::2 (2001:638:709:f::2)  0.052 ms  0.011 ms  0.008 ms
 3  2001:638:709:b::1 (2001:638:709:b::1)  0.117 ms  0.013 ms  0.010 ms


mininet> h2 traceroute 2001:638:709:a::1
traceroute to 2001:638:709:a::1 (2001:638:709:a::1), 30 hops max, 80 byte packets
 1  2001:638:709:b::e (2001:638:709:b::e)  0.033 ms  0.008 ms  0.007 ms
 2  2001:638:709:e::1 (2001:638:709:e::1)  0.054 ms  0.011 ms  0.009 ms
 3  2001:638:709:a::1 (2001:638:709:a::1)  0.025 ms  0.010 ms  0.010 ms
\end{minted}
\captionof{listing}{Asymetric route}
\label{listing:p3b}
\inputminted[linenos=true, frame=lines]{python}{src/p3.b.py}

\subsection*{c)}
We continue to run the mininet of Listing \ref{listing:p3b}, and we manually set
the mtu of the interface on r1(a.k.a r1-eth1) and r2(a.k.a r2-eth0), and using
ping6 to ping h2 from h1 with a package size of 1450.(default MTU is 1500 on h1)

\captionof{listing}{mininet result of c)}
\begin{minted}[linenos=true, frame=lines]{bash}
mininet> r1 ifconfig r1-eth1 mtu 1400
mininet> r2 ifconfig r2-eth0 mtu 1400
mininet> h1 ifconfig
h1-eth0: flags=4163<UP,BROADCAST,RUNNING,MULTICAST>  mtu 1500
        inet6 fe80::f8fb:27ff:fe5a:ef33  prefixlen 64  scopeid 0x20<link>
        inet6 2001:638:709:a::1  prefixlen 64  scopeid 0x0<global>
        ether fa:fb:27:5a:ef:33  txqueuelen 1000  (Ethernet)
        RX packets 71  bytes 7026 (7.0 KB)
        RX errors 0  dropped 0  overruns 0  frame 0
        TX packets 45  bytes 4282 (4.2 KB)
        TX errors 0  dropped 0 overruns 0  carrier 0  collisions 0

lo: flags=73<UP,LOOPBACK,RUNNING>  mtu 65536
        inet 127.0.0.1  netmask 255.0.0.0
        inet6 ::1  prefixlen 128  scopeid 0x10<host>
        loop  txqueuelen 1000  (Local Loopback)
        RX packets 12  bytes 1716 (1.7 KB)
        RX errors 0  dropped 0  overruns 0  frame 0
        TX packets 12  bytes 1716 (1.7 KB)
        TX errors 0  dropped 0 overruns 0  carrier 0  collisions 0
mininet> h2 ifconfig
h2-eth0: flags=4163<UP,BROADCAST,RUNNING,MULTICAST>  mtu 1500
        inet6 fe80::c041:eeff:fe30:8074  prefixlen 64  scopeid 0x20<link>
        inet6 2001:638:709:b::1  prefixlen 64  scopeid 0x0<global>
        ether c2:41:ee:30:80:74  txqueuelen 1000  (Ethernet)
        RX packets 74  bytes 7284 (7.2 KB)
        RX errors 0  dropped 0  overruns 0  frame 0
        TX packets 47  bytes 4454 (4.4 KB)
        TX errors 0  dropped 0 overruns 0  carrier 0  collisions 0

lo: flags=73<UP,LOOPBACK,RUNNING>  mtu 65536
        inet 127.0.0.1  netmask 255.0.0.0
        inet6 ::1  prefixlen 128  scopeid 0x10<host>
        loop  txqueuelen 1000  (Local Loopback)
        RX packets 12  bytes 1716 (1.7 KB)
        RX errors 0  dropped 0  overruns 0  frame 0
        TX packets 12  bytes 1716 (1.7 KB)
        TX errors 0  dropped 0 overruns 0  carrier 0  collisions 0

mininet> h1 tcpdump -i h1-eth0 -s 65535 -w ping1450mtu.dump &
tcpdump: h1 ping6 -M do -s 1450 2001:638:709:b::1
listening on h1-eth0, link-type EN10MB (Ethernet), capture size 65535 bytes
PING 2001:638:709:b::1(2001:638:709:b::1) 1450 data bytes
From 2001:638:709:a::f icmp_seq=1 Packet too big: mtu=1400
1458 bytes from 2001:638:709:b::1: icmp_seq=2 ttl=62 time=0.208 ms
1458 bytes from 2001:638:709:b::1: icmp_seq=3 ttl=62 time=0.094 ms
1458 bytes from 2001:638:709:b::1: icmp_seq=4 ttl=62 time=0.083 ms
1458 bytes from 2001:638:709:b::1: icmp_seq=5 ttl=62 time=0.085 ms
1458 bytes from 2001:638:709:b::1: icmp_seq=6 ttl=62 time=0.081 ms
1458 bytes from 2001:638:709:b::1: icmp_seq=7 ttl=62 time=0.084 ms
1458 bytes from 2001:638:709:b::1: icmp_seq=8 ttl=62 time=0.083 ms
^C40 packets captured
40 packets received by filter
0 packets dropped by kernel
\end{minted}
From the command line output, we can see that there is a message indicating that
the package is too big.

We import the dump file `ping1450mtu.dump` into the wireshark. We can clearly
observe that all the ping6 packets are using ICMPv6 protocol. The first echo
ICMPv6 packet has 1450 bytes of data and the first echo
request didn't receive the echo response.

However, right after the first echo packet, h1(2001:638:709:a::1) receive
a ICMPv6 packet from r1(2001:638:709:a::f)
indicating that the packet is too big, with MTU of 1400.

Afterwards, the echo ICMPv6 packet is transfered in 2 IPv6 packets, and we
can see that the ICMPv6 packet has been fragmented into 2 packets.
\subsection*{d)}
We continue from previous problem, runing the tracepath:
\captionof{listing}{Running tracepath}
\begin{minted}[linenos=true, frame=lines]{bash}
mininet> h1 tracepath -n 2001:638:709:b::1
 1?: [LOCALHOST]                        1.176ms pmtu 1500
 1:  2001:638:709:a::f                                     0.082ms
 1:  2001:638:709:a::f                                     0.016ms
 2:  2001:638:709:a::f                                     0.020ms pmtu 1400
 2:  2001:638:709:f::2                                     0.064ms
 3:  2001:638:709:b::1                                     0.150ms reached
     Resume: pmtu 1400 hops 3 back 3 1?: [LOCALHOST]                        0.552ms pmtu 1500
\end{minted}
Ping can determine the MTU value because if the current MTU is bigger, the
packet will be dropped in the route and an ICMPv6 will be forwarded back
including the error message and the supported MTU value.

Tracepath, just like traceroute\cite{slashroot.in}, using the TTL and the error ICMPv6 packet to
determine the right MTU value. It set different TTL value so that the packet
only perform limited hops. Using this technique and in combination of the error
packet sent back when MTU has exceeded and dropped, we can determine the link
that limit the MTU and the optimal MTU value.

\subsection*{e)}
No, we can do a experiment in the mininet
\captionof{listing}{Path MTU cache experiment}
\begin{minted}[linenos=true, frame=lines]{bash}
mininet> h1 tracepath -n 2001:638:709:b::1
 1?: [LOCALHOST]                        0.016ms pmtu 1500
 1:  2001:638:709:a::f                                     0.048ms
 1:  2001:638:709:a::f                                     0.015ms
 2:  2001:638:709:a::f                                     0.015ms pmtu 1400
 2:  2001:638:709:f::2                                     0.026ms
 3:  2001:638:709:b::1                                     0.037ms reached
     Resume: pmtu 1400 hops 3 back 3
mininet> h1 tracepath -n 2001:638:709:b::1
 1?: [LOCALHOST]                        0.016ms pmtu 1400
 1:  2001:638:709:a::f                                     0.042ms
 1:  2001:638:709:a::f                                     0.016ms
 2:  2001:638:709:f::2                                     0.025ms
 3:  2001:638:709:b::1                                     0.033ms reached
     Resume: pmtu 1400 hops 3 back 3
\end{minted}
As we can see, in the second tracepath, the pmtu value starts from 1400. Linux
has obviously cached the path MTU. Such a cache is stored in the routing table
entries. The host will not have routing table entries for every destination, but
it can cache per-host route for every active destination\cite{RFC1191}.


\bibliography{hw3.bib} 
\bibliographystyle{plain}

\end{document}
