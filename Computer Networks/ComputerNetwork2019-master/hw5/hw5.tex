\documentclass{article}
%% course name and homework number
\newcommand{\coursename}{Computer Network}
\newcommand{\hwnumber}{5}
\usepackage[utf8]{inputenc}
\usepackage{amsmath}
\usepackage{amssymb}
\usepackage{amsfonts}
\usepackage{amssymb}
\usepackage{minted}
\usepackage{graphicx}
\usepackage{algorithm}
\usepackage{algorithmic}
\graphicspath{ {img/} }
\usepackage{titlesec}
\usepackage[a4paper,margin=1in,footskip=0.25in]{geometry}
\usepackage{fancyhdr}
\usepackage{enumitem} %% custom auto-index for enumerate environment
\pagestyle{fancy}
%basic page layout

%draw finite state machine
\usepackage{tikz}
\newcommand{\Lcvy}{\mathcal{L}}
%header and footer settings
\lhead{\coursename \thinspace Assignment \hwnumber}
\chead{Yiping Deng}
\rhead{\today}

\titlelabel{\thetitle\enspace}

\begin{document}
\title{\coursename \thinspace Assignment \hwnumber}
\author{Yiping Deng}
\maketitle
\thispagestyle{fancy}

%% start of homework
\section*{Problem 1}
\subsection*{a)}
SEQ = 1030, ACK= 3848 F=ACK WIN=4000, it carries 1200 due to the
ackknowlegement.(2230 - 1030).

\subsection*{b)}
To inform the other side of the connection that the part of the data in the
buffer has been consumed, and the window size has changed since.

\subsection*{c)}
SACK works by sending the received segment range in the duplicate
acknowledgement. This will allow the other side of the transmission to send only
the missing segment, not all the segment after the lost segment.

The two number in the SACK indicates the left edge and the right edge of the
segment received. Since we cannot simply specify the segment missing, we can, in
the other words, specify the segments(ranges of the data) we received.

\subsection*{d)}
In segment 8, the ACK is used to indicate missing segments. Left edge is 3430,
right edge is 4630
\subsection*{e)}
The client goes to the wait state. Client initiated the closing(active close),
sending a FIN, receive a ACK and FIN, then send the ACK back and go to wait in
case ACK got lost in the flight.

\section*{Problem 2}
\subsection*{a)}
1300000/12 = 108,333 p/s
\subsection*{b)}
Minimum: 30000
Maximum: 300000
\subsection*{c)}
6 segments got lost and is not yet transmitted at t = 12
\subsection*{d)}
At the beginning of the TCP connection, a large number of groups of TCP
fragments are send from t=0.5 to t=1.4. However, a lot of them are lost and
retransmitted during t=1.6 to t=3.3. From t=3.3 onwards is the normal TCP communicartion.


\end{document}
